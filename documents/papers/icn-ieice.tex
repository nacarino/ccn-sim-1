\documentclass[8pt,twocolumn]{article}
\usepackage[english]{babel}

\usepackage{graphicx}
\usepackage{amsmath}
\usepackage{amssymb}
\usepackage{amsthm}
\usepackage{epsfig}
\usepackage{listings, color}
\usepackage{xcolor, colortbl}
\usepackage{url}
\usepackage{fullpage}

\DeclareGraphicsExtensions{.png}

\begin{document}
\twocolumn[
\centerline{\huge \bf On Network naming for mobility in ICN networks}
\medskip
\centerline{Jairo Eduardo L\'{o}pez$^{1}$, Takuro Sato$^{2*}$}
\medskip
\centerline{${}^{1,2}$ Graduate School of Information and Telecommunications
Studies, Waseda University}
\centerline{jairo at ruri dot waseda dot jp${}^{1}$, t-sato at waseda dot jp${}^{2}$, ${}^{*}$Fellow, IEEE}
\bigskip
]

\section{Abstract}
It has been known for some time that our current Internet infrastructure
requires an overhaul. Over the last 3 decades there have been attempts to solve
the main issues plaguing our current network infrastructure. A lot of the 
issues have been solved in ad-hoc ways, while others have been completely
ignored. Mobility, for example, has required the implementation of complex
architectures which have patched the problem, but not solved it due to the 
flat naming of our current network infrastructure. 

Among the new ideas for a clean-slate architecture, one of the most
prominent has been Information Centric Networking (ICN), which attempts to put
content as the primary source of network traffic. This architecture has been
gaining momentum and shows a lot of promise, seamingly supporting mobility 
out of the box, but we feel that it is failing to take into account one of the
biggest source of issues in our current architecture; naming. In this paper we
will talk about naming in networks and show via a simple simulation how most
of the mobility issues in ICN could be solved via a 3 level naming/resolution
system, leaving open the discussion of how best to implement these systems. 
 
\section{Introduction}
Information Centric Networking (ICN) is a clean-state network architecture
initially proposed by Van Jacobson which attempts to make content and not
hosts, the network's first class citizens
 \cite{Jacobson:2009:NNC:1658939.1658941}. This particular shift to a
content-oriented model is well founded in fact as most network analysis shows
that for 2013 about 66\% of all IP traffic was caused by video transmission.
Estimates put IP video tranmission at 79\% for 2018 \cite{cisco:2014:Online}.
Another fact that makes this architecture compelling is that mobile IP traffic
is currently at 3\% of the total IP traffic, but is estimated to increase 4
fold in 5 years. While still a smaller chunk of total IP traffic, about 53\% of
current mobile IP traffic is video and is also expected to increase in the 
following years \cite{ciscomobile:2014:Online}. Taking these facts into
account, it would seem beneficial for any future network architecture to
take the high demand for content and mobility trend into account.

The ICN architecture is still under heavy debate, meaning that there are a lot
of competing proposals. The variations between some proposals can be really 
subtle. The most important ICN proposals have been summarized in \cite{6563278}.

\section{Current status of mobility in ICN Approaches}
Due to the basic architecture of ICN, mobility, at least for the receiver, is
assured. Since data can only be obtained after a device has sent an Interest
Packet, mobility means that when a terminal changes network, the Interest 
packets have to be retransmitted. For any other scenario, the ICN network
architectuire attempts to be impartial to the stneger that of the leaser l 
The main goal is to decouple the named
content from it's actual location so that any node in the network can supply
a request for that content.  

For the publishing side, or for when you need to keeo a stared station beteec

\section{NDN}
\section{Implementation}
\section{Future work}

\section*{Acknowledgement}
This research was supported by a grant-in-aid from the High-Tech Research
Center Project of the Ministry of Education, Culture, Sports, Science and
Technology (MEXT) \cite{mext:2014:Online}, Japan.

\bibliographystyle{acm}
\bibliography{myrefs}

\end{document}
