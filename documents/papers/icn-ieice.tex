\documentclass[8pt,twocolumn]{article}
\usepackage[english]{babel}

\usepackage{graphicx}
\usepackage{amsmath}
\usepackage{amssymb}
\usepackage{amsthm}
\usepackage{epsfig}
\usepackage{listings, color}
\usepackage{xcolor, colortbl}
\usepackage{url}
\usepackage{fullpage}

\DeclareGraphicsExtensions{.png}

\begin{document}
\twocolumn[
\centerline{\huge \bf NDN something something for people}
\medskip
\centerline{Jairo Eduardo L\'{o}pez$^{1}$, Takuro Sato$^{2*}$}
\medskip
\centerline{${}^{1,2}$ Graduate School of Information and Telecommunications
Studies, Waseda University}
\centerline{jairo at ruri dot waseda dot jp${}^{1}$, t-sato at waseda dot jp${}^{2}$, ${}^{*}$Fellow, IEEE}
\bigskip
]

\section{Abstract}
We use ndnSIM \cite{ndn367} a ns3 \cite{riley:ns3} \cite{ns3:2014:Online}
module that implements Named Data Networking (NDN)
 \cite{Jacobson:2009:NNC:1658939.1658941}. We follow the advice by Dr. John
Day in checking out how NDN, and it's most famous implementation CCN, rack
up to the proposed issues for next generation architectures 
\cite{Day:2008:PNA:1349793}. We take a look at the routing aspect, particularly
Routing on Flat Labels \cite{Caesar06rofl:routing} and notice how it doesn't
add up and just implements a solution utilizing Moore's Law as a solution. We
propose utilizing a DHT style location-aware naming, similar to Azure's
Cubit \cite{WongSirer2008ApproximateMatching} utilizing nearest neighbor 
searching using near optimal hashing \cite{Andoni06near-optimalhashing} to 
improve routing capabilites without modifying the NDN architecture. Our results
show that I really don't know what to do at this moment. 

\section{NDN}
\section{Implementation}
\section{Future work}

\section*{Acknowledgement}
This research was supported by a grant-in-aid from the High-Tech Research
Center Project of the Ministry of Education, Culture, Sports, Science and
Technology (MEXT) \cite{mext:2014:Online}, Japan.

\bibliographystyle{acm}
\bibliography{myrefs}

\end{document}
